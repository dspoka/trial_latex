% This is "sig-alternate.tex" V1.9 April 2009
% This file should be compiled with V2.4 of "sig-alternate.cls" April 2009
%
% This example file demonstrates the use of the 'sig-alternate.cls'
% V2.4 LaTeX2e document class file. It is for those submitting
% articles to ACM Conference Proceedings WHO DO NOT WISH TO
% STRICTLY ADHERE TO THE SIGS (PUBS-BOARD-ENDORSED) STYLE.
% The 'sig-alternate.cls' file will produce a similar-looking,
% albeit, 'tighter' paper resulting in, invariably, fewer pages.
%
% ----------------------------------------------------------------------------------------------------------------
% This .tex file (and associated .cls V2.4) produces:
%       1) The Permission Statement
%       2) The Conference (location) Info information
%       3) The Copyright Line with ACM data
%       4) NO page numbers
%
% as against the acm_proc_article-sp.cls file which
% DOES NOT produce 1) thru' 3) above.
%
% Using 'sig-alternate.cls' you have control, however, from within
% the source .tex file, over both the CopyrightYear
% (defaulted to 200X) and the ACM Copyright Data
% (defaulted to X-XXXXX-XX-X/XX/XX).
% e.g.
% \CopyrightYear{2007} will cause 2007 to appear in the copyright line.
% \crdata{0-12345-67-8/90/12} will cause 0-12345-67-8/90/12 to appear in the copyright line.
%
% ---------------------------------------------------------------------------------------------------------------
% This .tex source is an example which *does* use
% the .bib file (from which the .bbl file % is produced).
% REMEMBER HOWEVER: After having produced the .bbl file,
% and prior to final submission, you *NEED* to 'insert'
% your .bbl file into your source .tex file so as to provide
% ONE 'self-contained' source file.
%
% ================= IF YOU HAVE QUESTIONS =======================
% Questions regarding the SIGS styles, SIGS policies and
% procedures, Conferences etc. should be sent to
% Adrienne Griscti (griscti@acm.org)
%
% Technical questions _only_ to
% Gerald Murray (murray@hq.acm.org)
% ===============================================================
%
% For tracking purposes - this is V1.9 - April 2009

\documentclass{sig-alternate}
# All packages and command definitions that need to go
# before \begin{document} in both the blind and not-blind versions go
# in this file

\usepackage{subfigure}
\newcommand{\ignore}[1]{}

\newcommand{\blind}[1]{}
\newcommand{\notblind}[1]{#1}

\begin{document}
% --- Author Metadata here ---
\conferenceinfo{SIGCSE}{'14 Someplace TBD}
\CopyrightYear{2014} % Allows default copyright year (20XX) to be over-ridden - IF NEED BE.
%\crdata{0-12345-67-8/90/01}  % Allows default copyright data (0-89791-88-6/97/05) to be over-ridden - IF NEED BE.
% --- End of Author Metadata ---

\title{Ontology Querying}


\notblind{
\author{Daniel Spokoyny,
College of Creative Studies Computer Science \\
University of California, Santa Barbara \\
dspoka@gmail.com
}
}

\maketitle
\begin{abstract}
This is where the abstract goes.  This is where the abstract
goes. This is where the abstract goes.  This is where the 
abstract goes.  This is where the abstract goes.  This is where 
the abstract goes.  This is where the abstract goes.  This is 
where the abstract goes.  This is where the abstract goes.  This 
is where the abstract goes.

\end{abstract}

% A category with the (minimum) three required fields
\category{K.3.2}{Computers and Education}{Computer and Information Science Education}

\terms{Design, Human Factors}

\keywords{Ontology, Semantic}


\section{Introduction}
This is the intro to the paper.   Here we describe what the whole paper is
about at a high level.  We give the reader some basic idea of why the paper
is addressing an important problem, and what it is, in brief, that we
have to say about it.   We really want the reader to be very interested,
so this paragraph has to be written especially well.

The rest of the paper is organized as follows.
Section~\ref{sec:Background} describes background that a reader needs
to know to appreciate our contribution.  A survey of related
work is either included here, or broken out into a separate section.
Section~\ref{sec:Results} describes our results, including a
description of our experimental measures, the hypotheses we were
testing, and the results of our data.  Finally,
Section~\ref{sec:ConclusionsAndFutureWork} presents a summary of our
conclusions and suggestions for future work.

 

\section{Background}\label{sec:Background}
Here is an example of a related work section and how to cite sources.

The benefits of SBL have been most explored in introductory
courses \cite{p55-gestwicki, p505-hendrix, p165-narayanan}, but have
also been applied to specialized topics, including Human Computer
Interaction \cite{p117-hundhausen} and Game Design
\cite{p64-estey}. Game design and HCI require a hybrid mastery of
artistic elements and programming abilities. The artistic qualities of
these disciplines naturally lend themselves to the benefits of SBL,
which has a long history in the arts.

Participation in courses featuring SBL has shown to give students a
greater confidence in their own project quality and programming
abilities, and creates a sense of community among CS
students~\cite{p887-bergin, p64-estey}. As a part of SBL, students are
given the opportunity to enhance and exhibit skills that may be
neglected in a traditional computer science education---students often
collaborate as a team, discuss competing solutions, and explain their
work to their peers~\cite{p505-hendrix}.  Working collectively not
only creates an environment where students can learn from each
other---increasing their technical skills, but they also gain valuable
communication skills~\cite{p165-narayanan}. SBL is a well-attuned
educational response to industry desires for more refined
interpersonal and team skills from Computer Science
graduates~\cite{p64-estey}.  While the application of SBL to CS is
still in its early stages, and has yet to be widely adopted, the
studies that have been done have shown promising results.



\section{Results}\label{sec:Results}
In this section, we present the data we collected this with that.
We also present an analysis of that other thing.

\subsection{This}

We collected data about this from students in \blind{CMPSC56, } a 2nd-year CS course at \blind{a large public research university}\notblind{UC Santa Barbara}.

[TODO: Say more about this.]

\subsection{That}

[TODO: Describe that]

\subsection{That other thing}

[TODO: Describe that other thing]




\section{Conclusions and Future Work}\label{sec:ConclusionsAndFutureWork}
We have presented results of a study of this, that and some other thing.
Here we recap the major results, and their signficance for human progress
towards self-actualization as a species.

Future work may include repeating the experiment conducted in this
paper over future offerings of the course, to lend even more
statistical support to our claims, as well as investigating some ideas
that we've come up with in the meantime that we think will be even
better.


\notblind{
\section{Acknowledgments}
%We would like to thank our anonymous reviewers for their thoughtful comments. 

The research described in this paper is part of a multi-university
project supported by the National Science Foundation under CPATH
Awards CNS 0939055, 0939017 and 0939157. We gratefully acknowledge the
contributions of the PIs on those awards, N. Hari Narayanan, and Dean
Hendrix of Auburn University, Christopher Hundhausen of Washington
State University, and Martha Crosby from the University of Hawaii at
Manoa, and their students.  We also thank the students from UC Santa
Barbara's course CMPSC56 from S11, W12, S12, as well as the
TAs Kyle Klein, Alex Pucher, and Jasen Hall, and the 
undergraduate code review moderators: 
Scott Bishop,
Aaron Dodson, 
Geoff Douglas,
Alex Hamstra, 
Alex Mousavi, 
Carina Rammelkamp, 
Leif Dreizler, 
and 
Alison Van Pelt. 
We are also grateful to Tiffany Sabado for her
administrative support of this work.




}


% The following two commands are all you need in the
% initial runs of your .tex file to
% produce the bibliography for the citations in your paper.

\bibliographystyle{abbrv}
\bibliography{paper.bib}  % paper.bib is the name of the Bibliography in this case
% You must have a proper ".bib" file
%  and remember to run:
% latex bibtex latex latex
% to resolve all references
%
% ACM needs 'a single self-contained file' for the bib!
%
%APPENDICES are optional
%\balancecolumns
%\balancecolumns % GM June 2007
% That's all folks!

\end{document}
