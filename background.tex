Here is an example of a related work section and how to cite sources.

The benefits of SBL have been most explored in introductory
courses \cite{p55-gestwicki, p505-hendrix, p165-narayanan}, but have
also been applied to specialized topics, including Human Computer
Interaction \cite{p117-hundhausen} and Game Design
\cite{p64-estey}. Game design and HCI require a hybrid mastery of
artistic elements and programming abilities. The artistic qualities of
these disciplines naturally lend themselves to the benefits of SBL,
which has a long history in the arts.

Participation in courses featuring SBL has shown to give students a
greater confidence in their own project quality and programming
abilities, and creates a sense of community among CS
students~\cite{p887-bergin, p64-estey}. As a part of SBL, students are
given the opportunity to enhance and exhibit skills that may be
neglected in a traditional computer science education---students often
collaborate as a team, discuss competing solutions, and explain their
work to their peers~\cite{p505-hendrix}.  Working collectively not
only creates an environment where students can learn from each
other---increasing their technical skills, but they also gain valuable
communication skills~\cite{p165-narayanan}. SBL is a well-attuned
educational response to industry desires for more refined
interpersonal and team skills from Computer Science
graduates~\cite{p64-estey}.  While the application of SBL to CS is
still in its early stages, and has yet to be widely adopted, the
studies that have been done have shown promising results.

